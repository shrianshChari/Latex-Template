\documentclass[11pt]{report}
\renewcommand{\thesection}{\arabic{section}}

\usepackage{amssymb} % For useful math symbols
\usepackage{enumitem} % For controlling how the enumeration behaves
\usepackage{amsmath} % For align environment

\usepackage{graphicx} % For inserting images
\graphicspath{ {./images/} }

\usepackage{geometry} % For changing the margins
\geometry{letterpaper, margin=1.5in, headheight = 14pt}

\usepackage{xcolor} % Creating/using colors for the background of code snippets
\definecolor{DBGGray}{gray}{0.15}
\definecolor{LBGGray}{gray}{0.9}

\usepackage{fancyhdr} % Styling the headers/footers of document

% Code snippets
\usepackage{minted}

% For drawing graphs for CS courses
\usepackage{tikz}

% For writing algorithms for CS courses
\usepackage{algorithm}
\usepackage{algpseudocode}

\def\dark{0} % 1 means enable dark mode, anything else means light mode

\ifx\dark\undefined
	% Light mode by default
	\newcommand{\TextColor}{black}
	\newcommand{\PageColor}{white}
	\usemintedstyle{one-dark, bgcolor=LBGGray, tabsize=4, autogobble}
\else
	\if\dark1
		% Dark mode
		\newcommand{\TextColor}{white}
		\newcommand{\PageColor}{black}
		\usemintedstyle{one-dark, bgcolor=DBGGray, tabsize=4, autogobble}
	\else
		% Light mode
		\newcommand{\TextColor}{black}
		\newcommand{\PageColor}{white}
		\usemintedstyle{one-dark, bgcolor=LBGGray, tabsize=4, autogobble}
	\fi
\fi

% Languages whose stylings you want in the document
\newminted{java}{}
\newminted{c}{}
\newminted{r}{}

\pagecolor{\PageColor}
\color{\TextColor}

\newcommand{\HomeworkTitle}{HWX - CLASS 420} % Put the name of the assignment here!
\newcommand{\HomeworkAuthor}{Shriansh Chari} % The author of the assignment (should be me)
\newcommand{\HomeworkDate}{\today}

\fancyhf{}

\pagestyle{fancy}
\fancyhead{}
\fancyhead[L]{\HomeworkAuthor}
\fancyhead[C]{\HomeworkTitle}
\fancyhead[R]{Page \thepage}
\fancyfoot{}

\title{\textbf{\HomeworkTitle}}
\author{\HomeworkAuthor}
\date{\HomeworkDate}

\begin{document}

\maketitle

\section*{Question 1}
\begin{enumerate}[label = \alph*.]
	\item \begin{javacode}
			public class SomeCode {
				public static void main(String[] args) {
					System.out.println("Here is some code I can write" +
						"in LaTeX!");
				}
			}
	      \end{javacode}

	\item \begin{ccode}
			#include <stdio.h>

			int main() {
				printf("Code can be written in different langs!");

				return 0;
			}
	      \end{ccode}

	\item \begin{rcode}
		      print("I've used it a lot for R homework")

		      # Though I'm not using enough syntax here to distinguish R from
		      # Python or Lua...
	      \end{rcode}
\end{enumerate}

\newpage

\section*{Question 2}

Notice the neat header on the top of the page!

It's also useful to have the page number up there so that it won't be isolated in the footer.

Hey look, a red-black tree!

\begin{center}
	\begin{tikzpicture}
		\tikzstyle{state} = [draw, circle]
		\node [state]{5}
		child {node [state, xshift=-0.75cm]{3}
				child [red] {node [state]{1}
						child [\TextColor] {node [state]{0}}
						child [\TextColor] {node [state]{2}}}
				child {node [state] {4}}}
		child {node [state, xshift=0.75cm]{10}
				child {node [state]{8}}
				child [red] {node [state]{12}
						child [\TextColor] {node [state] {11}}
						child [\TextColor] {node [state] {14}}}};
	\end{tikzpicture}

	I made this red-black tree using \Verb|tikz|.
\end{center}

\end{document}

